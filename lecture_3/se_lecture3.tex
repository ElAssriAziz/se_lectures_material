\documentclass[10pt]{beamer}
\usepackage{xmpincl}
\includexmp{license}
\usepackage[T1]{fontenc}
\usepackage[latin1]{inputenc}
\usepackage{ae}
\usepackage[french]{babel}
\usepackage{array, longtable}
\usepackage{tikz}
\usetikzlibrary{arrows,shapes,trees,fit,decorations.pathmorphing,backgrounds}

\DeclareGraphicsRule{*}{mps}{*}{}
\usetheme{Antibes}
\setbeamertemplate{navigation symbols}{}


\usepackage{listings}
\usepackage{verbatim}
\makeatletter
\newwrite\lstvrb@out
\newenvironment{code}{%
  \begingroup
  \@bsphack
  \immediate\openout\lstvrb@out\jobname.lst
  \let\do\@makeother\dospecials\catcode`\^^M\active
  \def\verbatim@processline{%
    \immediate\write\lstvrb@out{\the\verbatim@line}}%
  \verbatim@start}{%
  \immediate\closeout\lstvrb@out
  \@esphack
  \endgroup

  \begin{alertblock}{}
    \lstinputlisting[language=java]{\jobname.lst}
  \end{alertblock}}
\makeatother

\lstset{language=java, basicstyle=\small, commentstyle=\color{blue}\textrm}

% for printing
% \usepackage{pgfpages}
% \pgfpagesuselayout{2 on 1}[a4paper,border shrink=5mm]
% \pgfpagesuselayout{resize to}[a4paper,border shrink=5mm,landscape]

\title{Computer Science Introductory Course MSC - Software engineering}

\subtitle{Lecture 3: Design patterns}
\usepackage{verbatim}

\author[Pablo Oliveira]{Pablo Oliveira \texttt{<pablo@sifflez.org>}}
\institute{ENST}

\date{}

\begin{document}
\begin{frame}
  \titlepage
\end{frame}

\begin{frame}
  \frametitle{Outline}
  \tableofcontents
\end{frame}

\section{Introduction}
\subsection{What is a design pattern ?}
\begin{frame}
\frametitle{What is a design pattern ?}
\begin{itemize}
  \item Proposed by architect C. Alexander in 70ths.
  \item General reusable solution to a recurring problem.
  \item Must be adapted to each concrete case.
  \item Patterns allow to communicate complex principle using a common vocabulary.
  \item Describe software abstractions.
  \item Each programming language provides some patterns already included as idioms:
    \begin{itemize}
      \item In java : encapsulation, subclassing, etc...
    \end{itemize}
  \item Use design patterns wisely (sometimes they only clutter the problem), always adapt them 
    to your particular problem and context.
\end{itemize}
\end{frame}

\subsection{Categories of design patterns}
\begin{frame}
\frametitle{Categories of design patterns}
\begin{center}
\begin{tabular}{|c|c|c|}
\hline
\emph{creational}&\emph{structural}&\emph{behavioural}\\
\hline
builder  & adapter   & chain of responsability \\
factory  & bridge    & command                 \\
prototype& composite & interpreter             \\
singleton& decorator & iterator                \\
         & fa�ade    & mediator                \\
         & flyweight & memento                 \\
         & proxy     & observer                \\
         &           & state                   \\
         &           & strategy                \\
         &           & visitor                 \\
\hline
\end{tabular}
\end{center}
\end{frame}

\section{Common Design patterns}
\subsection{Iterator}
\begin{frame}
  \frametitle{Iterator (UML)}
  \begin{center}
    \includegraphics[height=0.7\paperheight]{Iterator2.1}
  \end{center}
\end{frame}
\begin{frame}[fragile, shrink]
  \frametitle{Iterator (Java)}
    \begin{code}
  class Vector implements Container {
    private Item[] elements;
    private int last = -1;
    Vector(int size){elements = new Item[size];}
    Item get(int pos) {return elements[pos];}
    int getLast() {return last;}
    void Append(Item i){elements[++last] = i;}
    Iterator CreateIterator()
      {return new VectorIterator(this);}
  }
  class VectorIterator implements Iterator {
    private Vector v;
    private int cursor;
    VectorIterator(Vector v) {this.v = v;First();}
    void First() {cursor = 0;}
    void Next() {cursor++;}
    boolean isDone() {return cursor == v.getLast();}
    Item CurrentItem() {return v.get(cursor);}
  }
\end{code}
\end{frame}

\subsection{Decorator}
\begin{frame}
  \frametitle{Decorator (UML)}
  \begin{center}
    \includegraphics[width=0.9\paperwidth]{Decorator.1}
  \end{center}
\end{frame}
\begin{frame}[shrink]
  \frametitle{Decorator Example (UML)} 
  \begin{center}
    \includegraphics[width=0.9\paperwidth]{DecoratorExample.1}
  \end{center}
\end{frame}
\begin{frame}[fragile, shrink]
  \frametitle{Decorator Example (Java)}
\begin{code} 
interface Server {
  response handleRequest(request req);
}
abstract class ServerDecorator implements Server {
  protected Server decoratedServer;
  ServerDecorator(Server s) {decoratedServer = s;}
}
class ServerLogger extends ServerDecorator {
  ServerLogger(Server s){super(s);}
  response handleRequest(request req) {
    logRequest(req);
    return decoratedServer.handleRequest(req);
  }
  void logRequest(request red) {
    System.out.println
      ("Server got request from " + req.from);
  }
}
\end{code}
\end{frame}
\subsection{Singleton}
\begin{frame}
  \frametitle{Singleton (UML)}
  \begin{center}
    \includegraphics[width=0.5\paperwidth]{Singleton.1}
  \end{center}
  \structure{The singleton pattern ensures that only a single instance of an object is ever created.}
\end{frame}

\begin{frame}[fragile, shrink]
  \frametitle{Singleton Example (Java)}
\begin{code} 
public class Singleton {
   private static ClassicSingleton instance = null;
   protected Singleton() {} // no instantiation
   public static ClassicSingleton getInstance() {
      if(instance == null) {
         instance = new Singleton();
      }
      return instance;
   }
}
\end{code}
\end{frame}

\subsection{Visitor}
\begin{frame}[shrink]
  \frametitle{Visitor (UML)}
  \begin{center}
    \includegraphics[width=\paperwidth]{Visitor4.1}
  \end{center}
  \structure{The visitor pattern decouples the iteration over a structure and the operations made during the iteration}
\end{frame}

\begin{frame}
  \frametitle{Visitor Example: Tree Visitor}
\begin{center}
\begin{tikzpicture}
  \node at(0,0) {-}
    child{node{5}}
    child{node{+}
      child{node{1}}
      child{node{2}}};
\end{tikzpicture}
\end{center}
\structure{We have a tree structure, and want to perform various algorithms on it. Each algorithm should be described in its own class...}
\end{frame}


\begin{frame}[fragile]
  \frametitle{Visitor Example (Java): Object Structure}
\begin{code} 
abstract class TreeNode {
  TreeNode left, right;
}
class PlusN extends TreeNode {
  void acceptVisitor(TreeVisitor v) 
    {v.visitPlus(this);}
}
class MinusN extends TreeNode {
  void acceptVisitor(TreeVisitor v) 
    {v.visitMinus(this);}
}
class IntegerN extends TreeNode {
  Integer value;
  void acceptVisitor(TreeVisitor v) 
    {v.visitInteger(this);}
}
\end{code}
\end{frame}


\begin{frame}[fragile, shrink]
  \frametitle{Visitor Example (Java): TreeVisitor}
\begin{code} 
interface TreeVisitor {
  int visitInteger(IntegerN i);
  int visitPlus(PlusN p);
  int visitMinus(MinusN m);
}

class ReduceVisitor extends TreeVisitor {
  Integer value;
  void visitInteger(IntegerN i)
    {value = i.value;}
  void visitPlus(PlusN p)
    {
      p.left.acceptVisitor(this);
      Integer first = value;
      p.right.acceptVisitor(this);
      value += first;
    }
  ...
}
\end{code}
\end{frame}

\subsection{Factory}
\begin{frame}[shrink]
  \frametitle{Factory (UML)}
  \begin{center}
    \includegraphics[width=0.8\paperwidth]{AbstractFactory.1}
  \end{center}
\end{frame}

\begin{frame}[fragile,shrink]
  \frametitle{Factory Example (Java)}
\begin{code} 
interface Button{};
interface TextBox{};

interface GUIFactory{
  public Button createButton();
  public TextBox createTextBox();
}
WindowsFactory implements GUIFactory{
  public Button createButton() 
    {return new WindowsButton();}
  public TextBox createTextBox() 
    {return new WindowsTextBox();}
}
class LinuxFactory implements GUIFactory {
  public Button createButton() 
    {return new LinuxButton();}
...}
class Application {
  public Application(GUIFactory factory){
    Button button = factory.createButton();
      button.paint();
    }
  public static void main(String args[]){
    if (onWindows())
       new Application(new WindowsFactory());
    else
       new Application(new LinuxFactory());
  }}
\end{code}
\end{frame}

\subsection{Proxy}
\begin{frame}[shrink]
  \frametitle{Proxy(UML)}
  \begin{center}
    \includegraphics[width=0.8\paperwidth]{Proxy.1}
  \end{center}
\end{frame}


\end{document}
