\documentclass{exercices}
\usepackage{ae, aeguill, graphicx}
\usepackage{fullpage}
\usepackage{xspace}
\usepackage{color}
\usepackage{amsmath, amssymb}
\usepackage{latexsym}
\usepackage{url}
\usepackage{tikz}
\usetikzlibrary{decorations}
\usetikzlibrary{decorations.fractals}
\usepackage{multicol}
\usepackage{verbatim}

\renewcommand{\|}{\url|}
\begin{document}
\sujet{INF 821: Project - Subway Itinerary Planner}

This project is based on an initial proposal by Tavleen Singh
(\texttt{tavleen.singh@telecom-paristech.fr}).

\section{Introduction}

In this project you will design an application to compute shortest
itineraries in Paris using the ``Metro'' subway network.

The requirements for this project are separated in two categories:
\begin{itemize}
  \item Core requirements.
  \item Bonus requirements.
\end{itemize}

Core requirements, capture the fundamental features that are
expected in the final deliverable; they are mandatory.
Bonus requirements provide extra niceties to the application; they
are optional.

Once \textbf{and only once all the mandatory requirements have
been specified and implemented}, you may choose to work on some bonus
requirements for extra credit.

\section{Core requirements}

Your application will display a graph of the Paris' metro network.
An interface, either graphic or textual, will allow the user to select
two locations (starting location and ending location).
The application will then compute the shortest route between these two
locations and highlight the found itinerary on the displayed graph.

When computing the shortest route you will take into account:
\begin{itemize}
\item the time when traveling on train between two stations on the same line (you may assume a
  constant train speed, but you must take into account the distance between stations).
\item the time used to transfer, when changing lines (you may assume this to be
  constant).
\end{itemize}

The Paris metro network data may be retrieved from :
\url{http://www.di.ens.fr/~granboul/enseignement/mmfai/algo2001-2002/tp7/metro+.plan}.
If you decide to use this source for your data, you \textbf{must} ask permission
to the original author to use it if you ever release publicly your application.

\section{Bonus requirements}

\subsection{Distributed ``friend'' locator (Easy)}
Add a web interface to your application that allow users to specify their
current position. A person can then search for the location of one of her
friends.

\subsection{Tourist sightseeing map (Easy)}
Modify your application, so users can input a list
of monuments (ex. Tour Eiffel, Louvre, Sacr\'e C\oe ur) and the application
finds the shortest path that goes through all the selected monuments.

\subsection{Generic subway extractor (Hard)}
Write an OSM XML parser that extract the subway nodes from
any \url{http://www.openstreetmap.org/} city map.
This should allow to easily import any city's subway
transportation system in your application.



\end{document}