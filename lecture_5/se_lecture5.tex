\documentclass[10pt]{beamer}
\usepackage{xmpincl}
\includexmp{license}
\usepackage[T1]{fontenc}
\usepackage[latin1]{inputenc}
\usepackage{ae}
\usepackage[french]{babel}
\usepackage{array, longtable}
\usetheme{Antibes}
\setbeamertemplate{navigation symbols}{}


\usepackage{listings}
\usepackage{verbatim}
\makeatletter
\newwrite\lstvrb@out
\newenvironment{code}{%
  \begingroup
  \@bsphack
  \immediate\openout\lstvrb@out\jobname.lst
  \let\do\@makeother\dospecials\catcode`\^^M\active
  \def\verbatim@processline{%
    \immediate\write\lstvrb@out{\the\verbatim@line}}%
  \verbatim@start}{%
  \immediate\closeout\lstvrb@out
  \@esphack
  \endgroup

  \begin{alertblock}{}
    \lstinputlisting[language=java]{\jobname.lst}
  \end{alertblock}}
\makeatother

\lstset{language=java, basicstyle=\small, commentstyle=\color{blue}\textrm}

% for printing
% \usepackage{pgfpages}
% \pgfpagesuselayout{2 on 1}[a4paper,border shrink=5mm]
% \pgfpagesuselayout{resize to}[a4paper,border shrink=5mm,landscape]

\title{Computer Science Introductory Course MSC - Software engineering}

\subtitle{Lecture 5: Testing}
\usepackage{verbatim}

\author[Pablo Oliveira]{Pablo Oliveira \texttt{<pablo@sifflez.org>}}
\institute{ENST}

\date{}

\begin{document}
\begin{frame}
  \titlepage
\end{frame}

\begin{frame}
  \frametitle{Outline}
  \tableofcontents
\end{frame}

\section{Introduction}
\begin{frame}
\frametitle{Introduction}
\begin{itemize}
  \item Verification and Validation:
    \begin{itemize}
    \item Validation ensures that the software fulfills the requirements.
    \item \alert{Verification} ensures that the software meets the
      specification, three approaches :
      \begin{itemize}
      \item Prove correctness by formal verification: costly, do not prevent
        from bugs in the specification.
      \item Code inspection by peer reviews.
      \item \alert{Testing}.
      \end{itemize}
    \end{itemize}
\end{itemize}
\end{frame}

\section{What to test ?}
\begin{frame}
  \frametitle{What to test ?}
  \begin{itemize}
    \item Running the program on all possible inputs is impossible for complex
      problems:
      \begin{itemize}
      \item exploration space might be insanely large (or worse infinite)
      \end{itemize}
    \item Test on a subset of inputs:
      \begin{itemize}
      \item Partition inputs in significant classes maximizing the coverage of all the possible cases.
      \item To do this choose particular inputs for your tests:
        \begin{itemize}
          \item inputs that tests all the control branches of your code
          \item boundary cases (detect overflow and off by one bugs)
          \item duplicate, null or invalid inputs.
        \end{itemize}
      \end{itemize}
  \end{itemize}
\end{frame}

\begin{frame}[fragile]
  \frametitle{Example of partitionning (1/2)}
\begin{verbatim}
specification:
    int compare (int a, int b);

    The function compare returns:
       0 if a is equal to b
       1 if a is strictly superior to b
      -1 if a is strictly inferior to b
\end{verbatim}

\structure{Q: What inputs would you test ?}
\end{frame}

\begin{frame}[fragile]
  \frametitle{Example of partitionning (2/2)}
\begin{verbatim}
    int compare (int a, int b) {
      int c = a-b;
      if (c == 0) return 0;
      else if (c<0) return -1;
      else return 1;
    }
\end{verbatim}
\pause
\begin{verbatim}
    System.out.println(compare(10,10));          ->  0
    System.out.println(compare(10,5));           ->  1
    System.out.println(compare(-10,-5));         -> -1
    System.out.println(compare(-2147483648,1));  ->  1
\end{verbatim}
\end{frame}

\section{Types of tests}
\begin{frame}[fragile]
  \frametitle{Black box and White box testing}
  \structure{Black box testing}
  \begin{itemize}
  \item Generate test cases from the specification only.
  \item Do not make the same assumptions than the programmer.
  \item Tests are independent of the implementation.
  \end{itemize}

  \structure{White box testing}
  \begin{itemize}
    \item Generate test cases from the source code.
    \item Improves coverage: we know the different control paths in the code.
  \end{itemize}
\end{frame}

\begin{frame}[fragile]
  \frametitle{Unit tests}
  \begin{itemize}
  \item A unit is the smallest testable part of an application.
  \item Test a single functionality in the code.
  \item Usually tests a single method.
  \item Unit tests allow to isolate the parts of the system and show they are
    correct.
  \item Most useful during the implemenation phase.
  \end{itemize}
\end{frame}

\begin{frame}[fragile]
  \frametitle{Functional tests}
  \begin{itemize}
    \item Functional tests verify the program as a whole.
    \item Centered in functionality which may be distributed among many classes
      and functions.
    \item Important during the integration phase.
  \end{itemize}
\end{frame}

\begin{frame}[fragile]
  \frametitle{Regressions tests}
  \begin{itemize}
    \item Each time a bug is detected, a test that catches it must be written.
    \item If later on code is changed, the test ensures that if the bug appears
      again, it will be catched.
  \end{itemize}
\end{frame}

\section{Automated testing}
\begin{frame}[fragile]
  \frametitle{JUnit}
  \begin{itemize}
    \item Allows automazing tests.
    \item Helps during regression testing.
    \item \verb!http://www.junit.org/!
  \end{itemize}
\end{frame}

\begin{frame}[fragile]
  \frametitle{Example(1/2)}
\begin{verbatim}
import junit.framework.*;

public class TestCompare extends TestCase {
  CompareClass comp;
  protected void setUp() {
    comp = new CompareClass();
  }
  public void testPositive() {
    int compare = comp.compare(10,5);
    assertEquals(compare, 1);
  }
  public void testBoundaries() {
    int compare = comp.compare(-2147483648,1);
    assertEquals(compare, -1);
  }
}
\end{verbatim}
\end{frame}

\begin{frame}[fragile,shrink]
  \frametitle{Example(2/2)}
\tiny
\begin{verbatim}
$ javac -cp junit-4.5.jar:. TestCompare.java
$ java -cp junit-4.5.jar:. junit.textui.TestRunner TestCompare
..F
Time: 0,003
There was 1 failure:
1) testBoundaries(TestIt)junit.framework.AssertionFailedError: expected:<1> but was:<-1>
	at TestCOmpare.testBoundaries(TestIt.java:11)
	at sun.reflect.NativeMethodAccessorImpl.invoke0(Native Method)
	at sun.reflect.NativeMethodAccessorImpl.invoke(NativeMethodAccessorImpl.java:39)
	at sun.reflect.DelegatingMethodAccessorImpl.invoke(DelegatingMethodAccessorImpl.java:25)

FAILURES!!!
Tests run: 2,  Failures: 1,  Errors: 0
\end{verbatim}
\end{frame}

\begin{frame}[fragile]

    \tiny

  \begin{center}
        This work is licensed under a Creative Commons Attribution-Noncommercial-Share Alike 3.0 Unported License.
    \vtop{
      \vskip-3ex
      \hbox{
        \includegraphics[height=0.4cm]{cc-logo}
      }
    }
  \end{center}
\end{frame}


\end{document}
